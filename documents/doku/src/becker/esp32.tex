\section{Einführung (muss nach vorne) (Becker, Specht)}

Für die Programmierung der zeitkritischen Funktionalitäten wurde ein ESP-32 verwenden, da Funktion wie das Fahren an strengere zeitlichere Anforderungen
gebunden sind, als mit einem General-Purpose Betriebssystem zu realisieren. 

Die Wahl fiel hierbei genau auf den klassichen ESP-32 Chip, da dieser den Bluetooth Classic Standard unterstützt welcher für die Ansteuerung des Dualshock4 Controllers notwending ist.
Des Weiteren ist aufgrund des weiten Verbreitung viel Material zu dieser Serie verfügbar was zu einer signifikant einfacheren Fehlersuche beiträgt.

Es wurde allerdings explizit darauf verzichtet den Controller mittels des Arduino Frameworks zu programmieren. Grund hierfür war vor allem die Größe
der entstehenden Kompilate. Wie durch spätere Kapitel hoffentlich zu erkennen, sind auf dem System viele Funktionalitäten beherbergt was bedeutet, dass der Speicherplatz begrenzt ist.
Daher wurde der Controller mittels des Espressif IoT Development Frameworks (ESP-IDF) programmiert. Diese Frameworks bietet ebenfalls eine Pletora an Treiber für verschienste Funktionen, ist jedoch für Espressif Controller
zugeschnitten und damit deutlich platzsparender. Die Programmierung selbst fand in der Sprache C statt.

\section{Dualshock4 Treiber (Becker)}

\subsection{Version 1: eigens entwickelter Treiber}

\subsection{Version 2: Treiber basierend auf der Bluepad32 Bibliothek}

\subsection{Testing}

\section{PWM Board Treiber (Becker)}

\section{Ansteuerung der Servo Motoren (Becker)}

\section{Ansteuerung der Flywheel Motoren (Becker, Koch, Wohlrab)}

\section{Integration (Becker, Specht)}