\section{Geschützplattform (Becker)}

Als Geschützplattform wird im folgenden der Unterbau des Geschützes bezeichnet,
welcher den Geschützarm mit der Lochplatte des Fahrzeugs verbindet. Diese Plattform besteht
aus zwei Teilen, beide sind 3D-gedruckt. 

Der erste Teil ist der Unterbau, eine zylindrische Form mit einer 
Bodenplatte, die Schraublöcher besitzt, um den Aufbau mit dem Fahrzeug 
zu verbinden. Diesen Teil haben wir aus dem vorherigen Projekt übernommen, da 
die Konstruktion bereits auf dem Fahrzeug verbaut war und eine Eigenkonstruktion
ähnlich aussehen würde. Der Unterbau ist so konstruiert, dass er Platz für eine
PCA9685 PWM-Treiberplatine bietet, die für die Ansteuerung der Servo Motoren verwendet wird.
Für die Befestigung der Platine waren ebenfalls bereits Bohrlöcher vorhanden. Zusätzlich bietet
der Unterbau Platz für einen MG996R Servo Motor, der für die Drehung des darüberliegenden
Aufbaus verantwortlich ist.

Der zweite Teil besteht aus der Abdeckung
des Unterbaus, diese besitzt zum einen Bohrlöcher für die Verbindung des Servo Motors,
welcher im Unterbau positioniert ist, und zum anderen Bohrlöcher für die Verbindung mittels
des Geschützarms. Auch diese Abdeckung war bereits gegeben, jedoch wurde sie vermessen und
in FreeCAD rekonstruiert, um die Bohrlöcher für die Verbindung mit dem Geschütz so zu setzen,
dass das Verbindugnsstück und das Magazin ohne weitere Anpassungen montiert werden konnte.

\section{Magazin und Verbindungsstück (Becker)}

Sowohl das Magazin als auch das Verbindungsstück wurden auf Basis einer Vorlage gedruckt. 

Das Verbindungsstück besitzt Platz für einen weiteren MG996R Servo Motor, der für die Neigung 
des Geschützes verantwortlich ist. 

Das Magazin bestand aus einem linken und einem rechten Teil, welche in der Vorlage zusammengeklebt wurden.
Wie bereits im Abschnitt <Geschützarm> beschrieben, wurde für die Verbindung mit dem Geschützarm
der ersten Version Verbindungsstücke an beide Teile des Magazins gedruckt, um mittles Schrauben beide Teile
verbinden zu können. Zusätzlich wurden Schraublöcher an beide Teile des Magazins konstruiert, um auch diese mittels Schrauben zusammenzufügen.
Mit der zweiten Version des Geschützarmes wurden die Verbindungsstücke entfernt, da der Geschützarm nun an das Magazin geklebt wurde.
Grund für diese Änderung war, dass es aufgrund der deutlich kleineren Dimension des Geschützarmes nicht möglich war Verdbindungsstücke mit Schraublöchern
zu konstruieren. 

Zuletzt wurde auch die Länge des Lauf etwas vergrößert, um den Geschützarm besser stützen zu können.

\section{Magazingewicht (Becker)}

Das Magazin des Geschützes besitzt Platz für 6 Nerf-Darts und ein Magazingewicht, welches sowohl dafür sorgen soll, 
dass die Darts bei großer Vibration des Fahrzeugs nicht aus dem Magazin fallen, als auch dafür, dass nach einem Schuss das 
nächste Geschoss nachrutschen kann. 

In einem ersten Schritt wurde die Vorlage für das Magazingewicht angepasst, indem der Projektname eingraviert wurde.

Das Gewicht besitzt außerdem eine Aussparung, diese Verhindert dass das Gewicht die Bewegung des Servo Motors einschränkt, welcher die Darts
bei der Schussabgabe in die Flywheel Motoren befördert. Ohne dieses Loch würde der Servo versuchen, das Gewicht in den Lauf zu schieben,
was zu Materialschäden, entweder am Motor oder am Geschütz führen würde.

Diese Aussparungs führte allerdings zu Problemen. Wenn die Platform sehr weit nach
hinten geneigt wurde, konnte es passieren, dass der hintere Teil des Gewichts nicht mehr schwer genug war, um die Nerfs-Darts
in den Lauf zu drücken. Dieses Problem wurden mit selektiven Infill gelöst, alle anderen Teil wurden standardmäßig mit 15\% Infill gedruckt,
im hinteren Teil des Gewichts wurden jedoch mit 100\% Infill verwendet. Dieses Vorgehen führte dazu dass das vorher erläuterte Problem nicht mehr auftrat.

\section{Halterungen Stormversorgung (Becker)}

\section{Mikrocontroller-Case (Becker)}