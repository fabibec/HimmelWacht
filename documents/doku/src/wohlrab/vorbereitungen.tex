\chapter{Vorbereitungen}

\section{Einrichten der IDE (Wohlrab)} 

Zunächst wurde die Entwicklungsumgebung (IDE) entsprechend den Anforderungen eingerichtet. 
Dazu gehörte die Installation relevanter Erweiterungen zur Unterstützung von SSH-Verbindungen und GitHub-Integration sowie zur Entwicklung in C und zur Arbeit mit LaTeX. 
Im nächsten Schritt wurde die Projektordnerstruktur angelegt, um eine übersichtliche und effiziente Organisation des Quellcodes und der zugehörigen Dateien zu gewährleisten.

\section{Erstellung der Teileliste (Wohlrab)}

Zur Vorbereitung der Umsetzung wurde eine detaillierte Teileliste erstellt, die alle in der Projektdefinition vorgesehenen Komponenten umfasst. 
Für jede Komponente wurde eine geeignete physische Ausführung recherchiert und dokumentiert. 
Darüber hinaus erfolgte eine Sichtung des vorhandenen Bestands, bei der die Verfügbarkeit der einzelnen Bauteile in der Liste festgehalten wurde. 
Für nicht vorhandene Komponenten wurden mögliche Bezugsquellen ermittelt und Bestellungen aufgegeben. 
Da ein Großteil der benötigten Komponenten bereits vor Ort vorhanden war, konnten die Projektkosten signifikant reduziert werden. 
So entstand finanzieller Spielraum, um bei den noch zu beschaffenden Bauteilen eine höhere Qualität sicherzustellen.
