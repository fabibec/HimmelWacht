\chapter{Projektplanung}

\section{Definiton (Wohlrab)}

Die anfänglichen Überlegungen zum Projekt beschäftigten sich mit dem grundlegenden Aufbau des Systems.
Dazu wurden zunächst alle Funktionen und die dafür benötigten Komponenten zusammengestellt.
Anschließend wurde überlegt, wie sich die verschiedenen Komponenten miteinander verbinden lassen und welche von welchen Mikrocontrollern gesteuert werden sollen.
Dabei hat sich eine Zweiteilung des Systems ergeben: 
\begin{itemize}
    \item Alle Motoren und Servos sollen mit einem ESP32 verbunden werden.
    Dieser sollte außerdem über eine Bluetooth-Verbindung die Steuerbefehle des PlayStation-Controllers entgegennehmen.
    \item Die gesamte Sensorik des Systems soll über einen Raspberry Pi 5 ausgewertet werden.
    Er sammelt und kombiniert die Informationen aus den Sensorwerten und verarbeitet sie mithilfe eines KI-Modells.
    Dabei sollen Ziele über die Kamera erkannt und ihre Position mithilfe der weiteren Sensoren bestimmt werden.
    Auf dieser Grundlage sollen Steuerungsbefehle für den ESP32 berechnet und an diesen weitergegeben werden.
    Das Video der Kamera soll zudem über einen Webserver dargestellt werden.
\end{itemize}

\section{GANTT-Diagram (Wohlrab)}

Zu Beginn wird eine vollständige Auflistung aller für das Projekt relevanten Komponenten erstellt. 
Aus diesen Komponenten werden die entsprechenden Arbeitspakete abgeleitet.
Im nächsten Schritt erfolgt eine fundierte Abschätzung des Umfangs der einzelnen Arbeitspakete. 
Mithilfe dieser Schätzung wird der zeitliche Aufwand für die Bearbeitung jedes Pakets ermittelt.
Zur besseren Übersichtlichkeit werden die zeitlichen Abhängigkeiten zwischen den Arbeitspaketen dargestellt. 
Diese Darstellung ermöglicht es, die Reihenfolge der Aufgaben und ihre Zusammenhänge zu visualisieren. 
Dadurch wird eine effiziente und systematische Arbeitsweise gewährleistet.
Die Arbeitspakete werden anschließend auf die Gruppenmitglieder verteilt, wobei darauf geachtet wird, dass die Gesamtaufwände innerhalb des Teams ausgewogen sind. 
Dies trägt dazu bei, die Arbeitslast gerecht zu verteilen und die Effektivität des Teams zu maximieren.
Zur besseren Differenzierung sind die verschiedenen Arten von Arbeitspaketen im Diagramm farblich gekennzeichnet.

\section{Lastenheft (Wohlrab)}

Das Lastenheft stellt ein zentrales Dokument im Projektmanagement dar, da es die Anforderungen und Ziele des Projekts formell festhält. 
Es dient der präzisen Definition des Projektumfangs und legt den Rahmen fest, innerhalb dessen das Projekt durchgeführt wird. 
Durch die detaillierte Beschreibung der zu erfüllenden Anforderungen und Zielsetzungen gibt das Lastenheft eine klare Orientierung und leitet alle nachfolgenden Tätigkeiten ab. 
Nach Abschluss des Projekts ermöglicht es zudem eine objektive Beurteilung des Projekterfolgs, da es als Grundlage für die Evaluation der erreichten Ergebnisse und der Zielerreichung dient.

\section{Plakat (Wohlrab)}

Das Plakat ist ein zentrales Element der visuellen Kommunikation des Projekts, da es die wesentlichen Inhalte auf ansprechende Weise veranschaulicht. 
Ziel ist es, das Interesse von Personen zu wecken und ihnen eine schnelle, aber umfassende Übersicht über das Projekt zu vermitteln. 
Der Fokus liegt dabei auf den Hauptfunktionen des Projekts. 
Eine vollständige Aufstellung aller Funktionen wird bewusst vermieden, um die Aufmerksamkeit auf die Schlüsselaspekte zu lenken. 
Das Plakat ist somit ein effektives Kommunikationsmittel, das die wichtigsten Informationen auf klare Weise darstellt.
